% Plantilla para documentos LaTeX para enunciados
% Por Pedro Pablo Aste Kompen - ppaste@uc.cl
% Licencia Creative Commons BY-NC-SA 3.0
% http://creativecommons.org/licenses/by-nc-sa/3.0/

\documentclass[12pt]{article}

% Margen de 1 pulgada por lado
\usepackage{fullpage}
% Incluye gráficas
\usepackage{graphicx}
% Packages para matemáticas, por la American Mathematical Society
\usepackage{amssymb}
\usepackage{amsmath}
% Desactivar hyphenation
\usepackage[none]{hyphenat}
% Saltar entre párrafos - sin sangrías
\usepackage{parskip}
% Español y UTF-8
\usepackage[spanish]{babel}
\usepackage[utf8]{inputenc}
% Links en el documento
\usepackage{hyperref}
\usepackage{fancyhdr}
\setlength{\headheight}{15.2pt}
\setlength{\headsep}{5pt}
\pagestyle{fancy}

\newcommand{\N}{\mathbb{N}}
\newcommand{\Exp}[1]{\mathcal{E}_{#1}}
\newcommand{\List}[1]{\mathcal{L}_{#1}}
\newcommand{\EN}{\Exp{\N}}
\newcommand{\LN}{\List{\N}}

\newcommand{\comment}[1]{}
\newcommand{\lb}{\\~\\}
\newcommand{\eop}{_{\square}}
\newcommand{\hsig}{\hat{\sigma}}
\newcommand{\ra}{\rightarrow}
\newcommand{\lra}{\leftrightarrow}

% Cambiar por nombre completo + número de alumno
\newcommand{\alumno}{Victor Ruiz - 2320012J}
\rhead{Entrega 1 - \alumno}

\begin{document}
\thispagestyle{empty}
% Membrete
% PUC-ING-DCC-IIC1103
\begin{minipage}{2.3cm}
\includegraphics[width=2cm]{img/logo.pdf}
\vspace{0.5cm} % Altura de la corona del logo, así el texto queda alineado verticalmente con el círculo del logo.
\end{minipage}
\begin{minipage}{\linewidth}
\textsc{\raggedright \footnotesize
Pontificia Universidad Católica de Chile \\
Departamento de Ciencia de la Computación \\
IIC2413 - Matemáticas Discretas \\}
\end{minipage}


% Titulo
\begin{center}
\vspace{0.5cm}
{\huge\bf Tarea 1}\\
\vspace{0.2cm}
\today\\
\vspace{0.2cm}
\footnotesize{2º semestre 2024 - Profesores Eduardo Bustos  - Christian Alvarez}\\
\vspace{0.2cm}
\footnotesize{\alumno}
\rule{\textwidth}{0.05mm}
\end{center}



\section*{Análisis de los datos}
% Descripción de los datos (tipos) encontrados en Archivo1.csv y Archivo2.csv
\subsection*{Descripción de los datos}
Los datos de los archivos \texttt{Archivo1.csv} y \texttt{Archivo2.csv} corresponden a 
datos referentes a la base de datos de una Universidad con sus cursos, alumnos, profesores, notas, etc.

% Archivo1.csv: Cohorte;Código Plan;Plan;Bloqueo;RUN;DV;Nombres;Apellido Paterno;Apellido Materno;Nombre Completo;Número estudiante;Mail Personal;Mail Institucional;Periodo curso;Sigla curso;Asignatura;Sección;Nivel;Calificación;Nota;Último Logro;Fecha Logro;Última Carga

% Cohorte: string, Per´ıodo de ingreso o re-ingreso del estudiante, no nulo
% C´odigo Plan: string, C´odigo del plan de estudios, no nulo
% Plan: string, Nombre del plan de estudios, no nulo
% Bloqueo: Boolean, Indica si el estudiante tiene alguna restricci´on (deuda), no nulo
% RUN del estudiante: int, Rol ´Unico Nacional, no nulo
% DV del estudiante: char, D´ıgito verificador del RUN, no nulo
% Nombres: string, todos los nombres propios, no nulo
% Apellido Paterno: string, Apellido Paterno, no nulo
% Apellido Materno: string, Apellido Materno, admite nulos
% Nombre completo: string, Nombres + Apellido Paterno+Apellido Materno, no nulo
% N´umero de estudiante: int, identifica en forma ´unica a un estudiante y su plan de estudio, no nulo
% mail personal: string, Correo electr´onico personal, admite nulos
% mail institucional: string, Correo electr´onico institucional @lamejor.cl, admite nulos
% Periodo curso: string, periodo en que se curs´o o est´a cursando la asignatura, no nulo
% sigla curso: string, c´odigo que identifica a una asignatura, no nulo
% curso: string nombre de la asignatura, no nulo
% Secci´on: secci´on de la asignatura, no nulo
% Nivel del curso: semestre al que pertenece la asignatura en la malla, no nulo
% Calificaci´on: string, resultado conceptual de la evaluaci´on obtenida, no nulo
% Nota: float 1.0 a 7.0 o vac´ıo, resultado num´erico de la evaluaci´on obtenida, puede contener nulo
% ´Ultimo logro: Nivel del curso m´as atrasado aprobado por el estudiante, no nulo
% Fecha Logro: string, per´ıodo en que se obtuvo el ´Ultimo logro, no nulo
% ´Ultima toma de ramos: Per´ıodo de la ´ultima inscripci´on de asignaturas, no nulo

% Archivo2.csv: RUN;DV;PATERNO;MATERNO;NOMBRES:;MAIL PERSONAL;MAIL INSTITUCIONAL;TELÉFONO;CONTRATO;JORNADA DIURNO;JORNADA VESPERTINO;DEDICACIÓN;GRADO ACADÉMICO;JERARQUÍA;CARGO

% RUN del profesor/administrativo: int, Rol ´Unico Nacional, no nulo
% DV del profesor/administrativo: char, D´ıgito verificador del RUN, no nulo
% mail personal: string, Correo electr´onico personal, admite nulos
% mail institucional: string, Correo electr´onico institucional @lamejor.cl, admite nulos
% tel´efono: int de 9 d´ıgitos, tel´efono personal, admite nulos
% contrato: string, tipo de contrato (fuul, part-time, hora, etc.), admite nulos
% jornada diurna: string, ”DIURNO”, admite nulos
% jornada vespertina: string, ”vespertino”, admite nulos
% dedicaci´on: string, horas del contrato, admite nulos
% grado acad´emico:string, grado acad´emico ”Bachiller”, ”Licenciado”, ”Doctor”, admite nulos
% jerarquia: String, jerarqu´ıa acad´emica, admite nulos
% cargo: string, cargo administrativo, admite nulos

\begin{itemize}
    \item \texttt{Archivo1.csv}:
    \begin{itemize}
        \item \texttt{Cohorte}: string, Período de ingreso o re-ingreso del estudiante, no nulo
        \item \texttt{Código Plan}: string, Código del plan de estudios, no nulo
        \item \texttt{Plan}: string, Nombre del plan de estudios, no nulo
        \item \texttt{Bloqueo}: Boolean, Indica si el estudiante tiene alguna restricción (deuda), no nulo
        \item \texttt{RUN del estudiante}: int, Rol Único Nacional, no nulo
        \item \texttt{DV del estudiante}: char, Dígito verificador del RUN, no nulo
        \item \texttt{Nombres}: string, todos los nombres propios, no nulo
        \item \texttt{Apellido Paterno}: string, Apellido Paterno, no nulo
        \item \texttt{Apellido Materno}: string, Apellido Materno, admite nulos
        \item \texttt{Nombre completo}: string, Nombres + Apellido Paterno+Apellido Materno, no nulo
        \item \texttt{Número de estudiante}: int, identifica en forma única a un estudiante y su plan de estudio, no nulo
        \item \texttt{mail personal}: string, Correo electrónico personal, admite nulos
        \item \texttt{mail institucional}: string, Correo electrónico institucional @lamejor.cl, admite nulos
        \item \texttt{Periodo curso}: string, periodo en que se cursó o está cursando la asignatura, no nulo
        \item \texttt{sigla curso}: string, código que identifica a una asignatura, no nulo
        \item \texttt{curso}: string nombre de la asignatura, no nulo
        \item \texttt{Sección}: sección de la asignatura, no nulo
        \item \texttt{Nivel del curso}: semestre al que pertenece la asignatura en la malla, no nulo
        \item \texttt{Calificación}: string, resultado conceptual de la evaluación obtenida, no nulo
        \item \texttt{Nota}: float 1.0 a 7.0 o vacío, resultado numérico de la evaluación obtenida, puede contener nulo
        \item \texttt{Último logro}: Nivel del curso más atrasado aprobado por el estudiante, no nulo
        \item \texttt{Fecha Logro}: string, período en que se obtuvo el Último logro, no nulo
        \item \texttt{Última toma de ramos}: Período de la última inscripción de asignaturas, no nulo
    \end{itemize}
    \item \texttt{Archivo2.csv}:
    \item \texttt{RUN del profesor/administrativo}: int, Rol Único Nacional, no nulo
    \item \texttt{DV del profesor/administrativo}: char, Dígito verificador del RUN, no nulo
    \item \texttt{mail personal}: string, Correo electrónico personal, admite nulos
    \item \texttt{mail institucional}: string, Correo electrónico institucional @lamejor.cl, admite nulos
    \item \texttt{teléfono}: int de 9 dígitos, teléfono personal, admite nulos
    \item \texttt{contrato}: Permite diferneciar entre profesores y el resto de personas.
    \item \texttt{jornada diurna}: string, ”DIURNO”, admite nulos
    \item \texttt{jornada vespertina}: string, ”vespertino”, admite nulos
    \item \texttt{dedicación}: string, horas del contrato, admite nulos
    \item \texttt{grado académico}:string, grado académico ”Bachiller”, ”Licenciado”, ”Doctor”, admite nulos
    \item \texttt{jerarquía}: String, jerarquía académica, admite nulos
    \item \texttt{cargo}: Permite disitnguir entre profesores y administrativos.
\end{itemize}
\newpage
\subsection*{}
\section*{Tipos de errores de datos detectados por el programa}
Se dividen en 2 grandes casos:

\begin{itemize}
    \item \texttt{Problemas de carácteres}: Se detectan problemas en los carácteres especiales.
    \item \texttt{Traspaso de datos}: Se detectan problemas en los datos originales, dado que algunos datos que deberían ser numéricos, son strings.
\end{itemize}

Además, no se considera problema, pero se dectectan todos los espacios en vacio: \texttt{''}. Para solucionar esto, se 
reemplazarán por \texttt{'x'}, cuando no se admitan nulos en los campos, y por \texttt{''} cuando se admitan nulos.


\subsection*{Forma de solución utilizada}

Para solucionar los problemas de carácteres, se creo el metódo privado de la clase \texttt{Bananer} del módulo \texttt{Bananer.php}	
llamado \texttt{reemplazarCaracteresEspeciales}, el cual recibe el array de datos y reemplaza los carácteres especiales por los pedidos. 

Para solucionar los problemas de traspaso de datos, se creó el metódo publico0 de la clase \texttt{Bananer} del módulo 
\texttt{Bananer.php} llamado \texttt{deteccionArray}, el cual recibe el array de datos y reemplaza los datos que deberían ser numéricos, pero son strings, por su respectivo valor numérico.

\section*{Nombre de los archivos de salida y explicación de su contenido}
% Los archivos de salida son: $tablas = ['persona.csv', 'cursos.csv', 'profesor.csv', 'administrativo.csv', 'notas.csv', 'estudiante.csv']$

\begin{itemize}
    \item \texttt{persona.csv}: Contiene los datos de los estudiantes y profesores, con los campos \texttt{RUN, DV, Nombres, "Apellido Paterno", "Apellido Materno", "Nombre completo", Teléfono, "mail personal", "mail institucional"}.
    \item \texttt{cursos.csv}: Contiene los datos de los cursos, con los campos \texttt{"sigla curso", curso, Secciones, "Nivel del curso"}.
    \item \texttt{profesor.csv}: Contiene los datos de los profesores, con los campos \texttt{RUN, DV, contrato, jornada, dedicación, "grado académico", jerarquía, cargo}.
    \item \texttt{administrativo.csv}: Contiene los datos de los administrativos, con los campos \texttt{RUN, DV, cargo}.
    \item \texttt{notas.csv}: Contiene los datos de las notas, con los campos \texttt{RUN,"Número de estudiante", "sigla curso", "Periodo curso", Calificación, Nota}.
    \item \texttt{estudiante.csv}: Contiene los datos de los estudiantes, con los campos \texttt{Cohorte, "Número de estudiante", Código Plan, "Último Logro", "Fecha Logro", "Última toma de ramos}.
\end{itemize}

\section*{Ejecución del programa}

Para ejecutar el programa, se debe correr el archivo \texttt{main.php} con el comando \texttt{php main.php}.
Luego, se crearan en el directorio actual los archivos \texttt{persona.csv, cursos.csv, profesor.csv, administrativo.csv, notas.csv, estudiante.csv}.
Después, el programa preguntará por cual consulta realizar, y se debe ingresar el número de la consulta deseada. \texttt{1 o 2}.
Finalmente, se mostrará el resultado de la consulta en consola.
% Fin del documento
\end{document}
